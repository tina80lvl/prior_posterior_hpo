\documentclass[times,specification,annotation]{itmo-student-thesis}

%% Опции пакета:
%% - specification - если есть, генерируется задание, иначе не генерируется
%% - annotation - если есть, генерируется аннотация, иначе не генерируется
%% - times - делает все шрифтом Times New Roman, собирается с помощью xelatex
%% - pscyr - делает все шрифтом Times New Roman, требует пакета pscyr.

%% Делает запятую в формулах более интеллектуальной, например:
%% $1,5x$ будет читаться как полтора икса, а не один запятая пять иксов.
%% Однако если написать $1, 5x$, то все будет как прежде.
\usepackage{icomma}

%% Один из пакетов, позволяющий делать таблицы на всю ширину текста.
\usepackage{tabularx}

%% Данные пакеты необязательны к использованию в бакалаврских/магистерских
%% Они нужны для иллюстративных целей
%% Начало
\usepackage{tikz}
\usetikzlibrary{arrows}
\usepackage{filecontents}
\usepackage[export]{adjustbox}
\begin{filecontents}{thesis.bib} // TODO fix
	@inproceedings{hutter2010automated,
		title={Automated configuration of mixed integer programming solvers},
		author={Hutter, Frank and Hoos, Holger H and Leyton-Brown, Kevin},
		booktitle={International Conference on Integration of Artificial Intelligence (AI) and Operations Research (OR) Techniques in Constraint Programming},
		pages={186--202},
		year={2010},
		organization={Springer},
		langid={english}
	}
\end{filecontents}



%% Указываем файл с библиографией.
\addbibresource{thesis.bib}

\begin{document}
	
	\studygroup{M3435}
	\title{Алгоритмы настройки гиперпараметров на основе объединения априорных и апостериорных знаний о задаче}
	\author{Смирнова Валентина Сергеевна}{Смирнова В.С.}
	\supervisor{Фильченков Андрей Александрович}{Фильченков А.А.}{к.ф.-м.н}{доцент факультета информационных технологий и программирования, Университет ИТМО}
	\publishyear{2020}
	%% Дата выдачи задания. Можно не указывать, тогда надо будет заполнить от руки.
	%% Срок сдачи студентом работы. Можно не указывать, тогда надо будет заполнить от руки.
	%% Дата защиты. Можно не указывать, тогда надо будет заполнить от руки.
	\defencedate{18}{июня}{2019}
	
	\secretary{Павлова О.Н.}
	
	
	%% Задание
	%%% Техническое задание и исходные данные к работе
	\technicalspec{Требуется разработать алгоритм настройки гиперпараметров на основе объединения априорных и апостериорных знаний о задаче}
	
	%%% Содержание выпускной квалификационной работы (перечень подлежащих разработке вопросов)
	\plannedcontents{В работе должна быть показана эффективность разработанного решения по сравнению с сущесствующими}
	
	%%% Исходные материалы и пособия 
	\plannedsources{\begin{enumerate}
			\item Efficient and robust automated machine learning / M. Feurer  // Advances in neural information processing systems. — 2015.
			\item Leite R., Brazdil P. Active Testing Strategy to Predict the Best Classification Algorithm via Sampling and Metalearning. // ECAI. — 2010. 
	\end{enumerate}}

	%% Аннотация
	%%% Цель исследования
	\researchaim{Разработать алгоритм настройки гиперпараметров на основе объединения априорных и апостериорных знаний о задаче}
	
	%%% Задачи, решаемые в ВКР
	\researchtargets{\begin{enumerate}
			\item изучить существующие решения поставленной задачи;
			\item предложить и реализовать новый алгоритм;
			\item провести эксперименты, показывающие эффективность решения.
	\end{enumerate}}

	%%% Использование информационных ресурсов Internet
	%%%\internetsources{1}
	
	%%% Использование современных пакетов компьютерных программ и технологий
	\addadvancedsoftware{Пакет \texttt{numpy} }{Не предусмотрено}
	\addadvancedsoftware{Пакет \texttt{pandas} }{Не предусмотрено}
	\addadvancedsoftware{Пакет \texttt{robo} }{Не предусмотрено}
	\addadvancedsoftware{Пакет \texttt{matplotlib} }{Не предусмотрено}
	
	%%% Краткая характеристика полученных результатов 
	\researchsummary{ Был предложен и реализован эффективный алгоритм, решающий поставленную задачу. }
	
	%%% Гранты, полученные при выполнении работы 
	\researchfunding{Отсутствуют}
	
	%%% Наличие публикаций и выступлений на конференциях по теме выпускной работы
	\researchpublications{\begin{enumerate}
			\item IX Конгресс Молодых Учёных 	
	\end{enumerate}}
	
	
	%% Эта команда генерирует титульный лист и аннотацию.
	\maketitle{Бакалавр}
	
	%% Оглавление
	\tableofcontents
	
	%% Макрос для введения. Совместим со старым стилевиком.
	\startprefacepage
	
	Найти и обучить эффективный алгоритм классификации – трудоёмкая задача, которая включает в себя настройку гиперпараметров. Кроме того, по конкретной задаче могут быть известны некоторые факты, которые можно использовать для настройки гиперпараметров, что может дать существенное преимущество перед случайной расстановкой тех же параметров. В работе предполагается автоматизировать настройку гиперпараметров на основе объединения априорных и апостериорных знаний о конкретной задаче.\par
	\textbf{Оптимизация гиперпараметров} — задача машинного обучения по выбору набора оптимальных гиперпараметров для обучающего алгоритма. Одни и те же виды моделей машинного обучения могут требовать различные предположения, веса или скорости обучения для различных видов данных. Эти параметры называются гиперпараметрами и их следует настраивать так, чтобы модель могла оптимально решить задачу обучения. Для этого находится кортеж гиперпараметров, который даёт оптимальную модель, оптимизирующую заданную функцию потерь на заданных независимых данных.

	
	%% Начало содержательной части.
	%% Так помечается начало обзора.
	\chapter{Обзор существующих решений}
	\startrelatedwork
	
	\section{Определения и ключевые понятия}
	Введем определения и ключевые понятия, которые будут использоваться в дальнейшем.
	\begin{itemize}
		\item классификатор -- параметризованный алгоритм, решающий задачу классификации
		\item конфигурация -- фиксированный набор параметров классификатора
		\item алгоритм -- пара из классификатора и его конфигурации
		\item решаемая задача -- алгоритм с настроенными гиперпараметрами на конкретном датасете
		\item решённая задача -- алгоритм с оптимизированными гиперпараметрами на конкретном датасете
		\item текущее решение (текущая задача) -- решаемая задача, для которой могут использоваться сведения из решённых задач
	\end{itemize}
	\chapterconclusion
	
	
	\chapter{Предложенное решение}
	\chapterconclusion
	
	\chapter{Анализ полученных результатов}
	\chapterconclusion
	
	
	%% Макрос для заключения. Совместим со старым стилевиком.
	\startconclusionpage
	В данном разделе размещается заключение.
	
	\printmainbibliography
	
	
	%% После этой команды chapter будет генерировать приложения, нумерованные русскими буквами.
	%% \startappendices из старого стилевика будет делать то же самое
	\appendix
	\chapter{Картиночки}
	Лалала
	
\end{document}